\documentclass[12pt, letterpaper]{article}
\usepackage{graphicx}
\usepackage[a4paper, total={6in, 10in}]{geometry}
\usepackage{amsmath}
\usepackage{bm}
\usepackage{listings}
\usepackage{xcolor}
\lstset { %
    language=C++,
    backgroundcolor=\color{black!5}, % set backgroundcolor
    basicstyle=\footnotesize,% basic font setting
}

\title{Distributed Systems Notes}
\author{Sam Kirk} 
\date{July 2024}

\begin{document}
\maketitle

\section*{Week 1}
Acronym Guide 
\begin{itemize}
    \item \textbf{RMI} - Remote Method Invocation
    \item \textbf{IPC} - Inter-Proccess Communication
    \item \textbf{MPI} - Message Passing Interface 
    \item \textbf{HPC} - High performance computing
    \item \textbf{TCP} - Transmission Control Protocol
\end{itemize}

\section*{Week 2}
Java: Write once, run everywhere 

\subsection*{Test driven development}
JUnit is a test framework. 
Makes test driven development easy.
Add test methods directly into code.

\begin{lstlisting}
    import static org.junit.jupiter.api.Assertions.assertEquals;
    import example.util.Calculator;
    import org.junit.jupiter.api.Test;

    class MyFirstTests {
        private final Calculator calculator = new Calculator();

        @Test 
        void addition() {
            assertEquals(2, calculator.add(1, 1));
        }
    }
\end{lstlisting}

\subsection{Serialisation}
Any object moving between vitual machines must be serialisable.
So when sending bytes along the network 
the class needs to be packed up and sent as zeros and ones.
User defined classes are not serialisable by default.
Can be done using JSON, XML and Protocol Buffers (Google).

\subsection*{Threads}
Concurrency within the one process.
No need to switch memory spaces so its 
much faster and locality is preserved. 
Data is shared between threads of execution 
No costly IPC (Inter-process Communication), but 
it does require coordination. So there is 
not isolation between threads which means 
one thread can corrupt entire process. 

See lecture 2, 1:37 for simple thread code example. 

\section*{Assignment 1}
We are building a client-server model, where the the calculator server 
provides computation services, and the client requests these services.

RMI is a Java API that allows an object running in one Java Virtual 
Machine (JVM) to invoke methods on an object running in another JVM.

An RMI Registery is a simple naming service that allows clients to 
obtain a reference to a remote object. It acts as a directory for locating remote objects.

\subsection*{Remote Interface}
The remote interface (Calculator.java) defines the methods that 
can be called remotely. It extends java.rmi.Remote.  

\subsection*{Remote Object Implementation}
The class implementing the remote interface (CalculatorImplementation.java) 
provides the actual implementation of the methods defined in the remote interface. 
This class should extend UnicastRemoteObject and implement the remote interface.

\subsection*{Server Program}
The server program (CalculatorServer.java) creates an instance of the remote object 
implementation, binds it to the RMI registry, and waits for client requests.

\subsection*{Client Program}
The client program (CalculatorClient.java) looks up the remote 
object in the RMI registry and invokes methods on it.

\subsection*{Stub}
Think of a stub as a "helper" on the client side. When a client wants to call a 
method on a remote object (an object that exists on a different computer), the 
client actually talks to the stub instead of the remote object directly. The stub 
acts like a placeholder for the real object. It takes the client's request, packages 
it up, sends it over the network to the server where the real object lives, and then 
sends the response back to the client. So, the stub "stands in" for the remote object 
and makes it seem like the object is local to the client.

\subsection*{Proxy}
A proxy is a broader term that means something similar to a stub but can be used in 
different contexts, not just in RMI. In general, a proxy is an object that controls 
access to another object. In the case of RMI, the stub acts as a proxy for the remote 
object because it controls access to it, managing the network communication between 
the client and the server.

\subsection*{Key Java Concepts}
\textbf{Interface:} In Java, an interface is a reference type, similar to a class, 
that can contain only constants, method signatures, default methods, static methods, 
and nested types. The methods in interfaces are abstract, meaning they don't have a 
body and must be implemented by the classes that implement the interface.

\textbf{Extends:} The extends keyword is used to indicate that a class is 
inheriting from another class. In Java RMI, interfaces extend the Remote interface 
to indicate that they can be called remotely.

\textbf{Implements:} A class uses the implements keyword to implement an interface.
This means the class provides concrete implementations for the methods defined in the interface.

\textbf{Throws:} In Java, the throws keyword is used in method signatures to 
indicate that the method may throw certain exceptions. For example, remote methods 
in RMI must declare that they can throw RemoteException, which indicates a 
communication-related exception that can occur during a remote method call.

\textbf{RemoteException:} A checked exception that indicates an issue during a 
remote method call, such as network problems. Methods in a remote interface must declare this exception.

\subsection*{File Explanations}
\textbf{Calculator.java:} This is the remote interface that defines the methods the client 
can call remotely. It extends Remote and all methods throw RemoteException.

\textbf{CalculatorImplementation.java:} This class implements the Calculator interface. 
It extends UnicastRemoteObject to make it a remote object. The constructor must
throw RemoteException, and all methods must provide implementations for the interface methods.

\textbf{CalculatorServer.java:} This is the server program. It creates an instance of the 
remote object implementation, registers it with the RMI registry under the name 
"CalculatorService," and waits for client requests.

\textbf{CalculatorClient.java:} The client program locates the RMI registry, 
looks up the remote object by its name ("CalculatorService"), and invokes methods on it.

\subsection*{More info about the server}
\textbf{String[] args in the main Method:} In Java, the main method serves as the 
entry point for any Java application. The String[] args parameter allows you to
 pass command-line arguments to the program. Although in this example, args is not 
 used, it's a standard part of the method signature. If you wanted to allow the 
 user to specify the port number or other configurations at runtime, 
 you could use args to parse these command-line arguments.

\textbf{Why Use try and catch?} In Java, the try block is used to wrap code that 
might throw an exception. Exceptions are events that disrupt the normal flow of the 
program, such as network errors, I/O errors, etc. The catch block is used to handle 
these exceptions. If an exception occurs within the try block, the code in the 
catch block is executed. This helps in preventing the program from crashing and 
allows you to handle the error gracefully.

In the context of Java RMI, many methods can throw RemoteException or 
other checked exceptions that need to be handled, hence the use of try and catch.

\textbf{Port 1099 and Its Significance:} Port 1099 is the default port number 
for the RMI registry in Java. The RMI registry is a service that allows clients to 
look up remote objects by name and obtain references to them. Using a specific 
port (1099 by default) helps standardize where the RMI registry can be found. 
However, you can choose a different port if needed, as long as both the 
server and client know which port to use.

\textbf{What registry.rebind Does:} The registry.rebind(String name, Remote obj) 
method binds the specified name to the specified remote object in the RMI 
registry. If an object with the same name is already bound in the registry, 
it is replaced with the new object. In the example: registry.rebind("CounterService", counter);
This line registers the counter remote object with the name "CounterService" in the 
RMI registry. This name is used by clients to look up the remote object.

\textbf{How the catch Block Works:} The catch block is executed if an exception occurs within 
the associated try block. It captures the exception and allows you to handle it.
In catch (Exception e) {e.printStackTrace();}, the catch block will catch the exception 
(of type Exception) and print the stack trace, which provides details about the error.

\subsection*{Compiling and Testing}
\begin{enumerate}
    \item Compile the Java files: javac *.java
    \item Start the RMI registry: rmiregistry
    \item (Could use rmiregistry (port number) \& which runs the process in the background)
    \item Run the server: java CalculatorServer
    \item Run the client: java CalculatorClient
\end{enumerate}
Actually, after a lot of trying, the CalculatorServer actually includes LocateRegistry.createRegistry(2099). 
This means that you don't have to manually start the rmiregistry. So I think you can just compile the files, 
then run java CalculatorServer, then java CalculatorClient. 

\subsection*{Port issues}
If port 1099 is already in use, run command: lsof -i :1099, to get info. Then run kill (PID) to stop. 
Can run: ps aux | grep java, to see all java action. 






\end{document}